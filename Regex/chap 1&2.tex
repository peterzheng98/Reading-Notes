% Homework template for Inference and Information
% UPDATE: September 26, 2017 by Xiangxiang
\documentclass[a4paper]{article}
\usepackage{ctex}
\ctexset{
proofname = \heiti{证明}
}
\usepackage{amsmath, amssymb, amsthm}
% amsmath: equation*, amssymb: mathbb, amsthm: proof
\usepackage{moreenum}
\usepackage{mathtools}
\usepackage{url}
\usepackage{bm}
\usepackage{enumitem}
\usepackage{graphicx}
\usepackage{subcaption}
\usepackage{booktabs} % toprule
\usepackage[mathcal]{eucal}
\usepackage[thehwcnt = 1]{iidef}
\usepackage[left = 2cm, right = 2cm, top = 3.5cm, bottom = 1.5cm]{geometry}
\newCJKfontfamily\BoldFont{STHeiti}
\thecourseinstitute{正则表达式入门经典}
\thecoursename{读书笔记}
\theterm{}
\hwname{第一章\ \ 正则表达式概述\ \ ;第二章\ \ 正则表达式工具和使用方法}
\slname{\heiti{解}}
\begin{document}
\courseheader
\name{\text{Peter Zheng}}

\begin{enumerate}
\setlength{\itemsep}{3\parskip}
\item 正则表达式:用于匹配文本中的字符序列的字符模式。(像编程语言但不全是,需要一个环境) \\
用途:匹配特定字符串是否包含允许的的字符模式(类似于Qt中的Mask)以及查找对应的字符序列。\\
举例:查找重复的单词,检查表单输入是否合法,转换日期格式,检查错误拼写等等\\
\BoldFont 作用:判断一个字符串是否与模式串匹配。\\

\item 使用正则表达式的分析方法:\\
1. 用自然语言表达说明意图。\\
2. 考察数据源以及其可能的内容。\\
3. 考察可以使用的正则表达式的选项。\\
4. 考虑灵敏度和特殊性。\\
5. 
\end{enumerate}

\end{document}
\begin{equation}
\end{equation}

%%% Local Variables:
%%% mode: late\rvx
%%% TeX-master: t
%%% End:
